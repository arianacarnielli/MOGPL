%%%%%%%%%%%%%%%%%%%%%%%%%%%%%%%%%%%%%%%%%%%%%%%%%%%%%%%%%%%%%%%%%%%%%%%%%%%%%%%%%%%%%
% PACOTES                                                                           %
%%%%%%%%%%%%%%%%%%%%%%%%%%%%%%%%%%%%%%%%%%%%%%%%%%%%%%%%%%%%%%%%%%%%%%%%%%%%%%%%%%%%%
\documentclass[a4paper,11pt]{amsart}

%-----------------------------------------------------------------------------------%
% LAYOUT DA PÁGINA                                                                  %
%-----------------------------------------------------------------------------------%
\usepackage[top=2.25cm, bottom=2.25cm, left=2.25cm, right=2.25cm]{geometry}
%\usepackage{fancyhdr} % Permite controlar como são exibidos os cabeçalhos

%-----------------------------------------------------------------------------------%
% FORMATAÇÃO DO TEXTO                                                               %
%-----------------------------------------------------------------------------------%
%\usepackage{setspace} % Permite definir o espaçamento entre linhas

%-----------------------------------------------------------------------------------%
% PACOTES DE IMAGENS                                                                %
%-----------------------------------------------------------------------------------%
\usepackage[pdftex]{graphicx}
\pdfsuppresswarningpagegroup=1 % A warning issued when several PDF images are
% imported in the same page. Mostly harmless, can be almost always supressed.
%\usepackage[pstarrows]{pict2e} % Amplia as funcionalidades do ambiente picture
\usepackage{tikz}
\usetikzlibrary{shapes, arrows, arrows.meta}

%-----------------------------------------------------------------------------------%
% PACOTES DE TABELAS                                                                %
%-----------------------------------------------------------------------------------%
\usepackage{array} % Facilita a formatação de tabelas
%\usepackage{multirow} % Permite criar células que ocupam várias linhas em uma tabela
\usepackage{longtable} % Permite criar tabelas que quebram de página

%-----------------------------------------------------------------------------------%
% PACOTES MATEMÁTICOS DE BASE                                                       %
%-----------------------------------------------------------------------------------%
\usepackage{amsfonts,amstext,amscd,bezier,amsthm,amssymb}
\usepackage[centertags]{amsmath}

%-----------------------------------------------------------------------------------%
% PACOTES DE SÍMBOLOS MATEMÁTICOS                                                   %
%-----------------------------------------------------------------------------------%
\usepackage{mathtools} % Símbolos matemáticos extras. (ex.: \xrightharpoon)
%\usepackage[integrals]{wasysym} % Muda o estilo das integrais, além de outros
%                                 símbolos extras
%\usepackage[nice]{nicefrac} % Permite o uso de frações "melhores". Usar \nicefrac{}{}

%-----------------------------------------------------------------------------------%
% PACOTES DE FONTES MATEMÁTICAS                                                     %
%-----------------------------------------------------------------------------------%
%\usepackage{mathbbol} % Quase todos os símbolos com \mathbb
%\usepackage{bbm} % Extensão dos símbolos de \mathbb. Usar comando \mathbbm
%\usepackage{calrsfs} % Muda o estilo de \mathcal
%\usepackage[mathcal]{euscript} % Muda o estilo de \mathcal

%-----------------------------------------------------------------------------------%
% PACOTES DE CODIFICAÇÃO DE FONTES                                                  %
%-----------------------------------------------------------------------------------%
\usepackage[utf8]{inputenc} % Permite o uso de caracteres ISO 8859-1, incluindo os
%                               caracteres acentuados diretamente.
\usepackage[T1]{fontenc} % Uso de fontes T1, necessário para tratar caracteres
%                          acentuados como um único bloco.

%-----------------------------------------------------------------------------------%
% PACOTES DE LÍNGUAS                                                                %
%-----------------------------------------------------------------------------------%
\usepackage[french]{babel} % Seleciona a língua do documento, definindo nomes de
%                              seções, nome do índice, da bibliografia, etc. Em caso
%                              de documento com mais de uma língua, a padrão é a
%                              última.
\NoAutoSpaceBeforeFDP % Utilizar em francês se quiser evitar espaços antes de :

%-----------------------------------------------------------------------------------%
% PACOTES DE BIBLIOGRAFIA                                                           %
%-----------------------------------------------------------------------------------%
%\usepackage{babelbib} % Permite definir a língua das entradas da bibliografia. Usar
%                       [fixlanguage] para uma mesma língua para todas as entradas e
%                       \selectbiblanguage{} para definir a língua. Um estilo compa-
%                       tível com babelbib deve ser usado (ex: babplain)
\usepackage{cite} % Organiza os elementos citados dentro de um mesmo \cite.

%-----------------------------------------------------------------------------------%
% PACOTES DE FONTES                                                                 %
%-----------------------------------------------------------------------------------%
% Computer Modern (fonte padrão)                                                    %
% - - - - - - - - - - - - - - - - - - - - - - - - - - - - - - - - - - - - - - - - - %
%\usepackage{ae} % A usar com a fonte padrão do LaTeX quando forem gerados PDFs, para
%                 corrigir erros de visualização

% Computer Modern Bright (sans serif)                                               %
% - - - - - - - - - - - - - - - - - - - - - - - - - - - - - - - - - - - - - - - - - %
%\usepackage{cmbright}

% Times New Roman                                                                   %
% - - - - - - - - - - - - - - - - - - - - - - - - - - - - - - - - - - - - - - - - - %
%\usepackage{mathptmx} % Muda texto e modo matemático
%\usepackage{times} % Apenas texto, não muda modo matemático

% Arial                                                                             %
% - - - - - - - - - - - - - - - - - - - - - - - - - - - - - - - - - - - - - - - - - %
%\usepackage[scaled]{uarial} % Arial como fonte sans serif padrão

% Palatino                                                                          %
% - - - - - - - - - - - - - - - - - - - - - - - - - - - - - - - - - - - - - - - - - %
%\usepackage{mathpazo} % Muda texto e modo matemático
%\usepackage{palatino} % Apenas texto, não muda modo matemático

% Concrete                                                                          %
% - - - - - - - - - - - - - - - - - - - - - - - - - - - - - - - - - - - - - - - - - %
%\usepackage{ccfonts} % Texto: Concrete; Matemático: Concrete Math
%\usepackage{ccfonts, eulervm} % Texto: Concrete; Matemático: Euler

% Iwona                                                                             %
% - - - - - - - - - - - - - - - - - - - - - - - - - - - - - - - - - - - - - - - - - %
%\usepackage[math]{iwona} % Texto e modo matemático: Iwona

% Kurier                                                                            %
% - - - - - - - - - - - - - - - - - - - - - - - - - - - - - - - - - - - - - - - - - %
%\usepackage[math]{kurier} % Texto e modo matemático: Kurier

% Antykwa Póltawskiego                                                              %
% - - - - - - - - - - - - - - - - - - - - - - - - - - - - - - - - - - - - - - - - - %
%\usepackage{antpolt} % Texto: Antykwa Póltawskiego; Matemático: nenhum
                     % Usar fontenc = QX ou OT4

% Utopia                                                                            %
% - - - - - - - - - - - - - - - - - - - - - - - - - - - - - - - - - - - - - - - - - %                     
%\usepackage{fourier} % Texto: Utopia; Matemático: Fourier

% KP Serif                                                                          %
% - - - - - - - - - - - - - - - - - - - - - - - - - - - - - - - - - - - - - - - - - %
\usepackage{kpfonts}

%-----------------------------------------------------------------------------------%
% CORES                                                                             %
%-----------------------------------------------------------------------------------%
\usepackage{color}
\definecolor{darkgreen}{rgb}{0,0.5,0}
\definecolor{darkmagenta}{rgb}{0.5,0,0.5}
\definecolor{darkgray}{rgb}{0.5,0.5,0.5}
\definecolor{darkblue}{rgb}{0.2,0.2,0.4}
\definecolor{darkred}{rgb}{0.6,0.15,0.15}
\definecolor{gray}{rgb}{0.65,0.65,0.65}
\definecolor{lightgray}{rgb}{0.8,0.8,0.8}
\definecolor{lightblue}{rgb}{0.5,0.5,1}
\definecolor{lightgreen}{rgb}{0.5,1,0.5}
\definecolor{deadred}{rgb}{0.7, 0.2, 0.2}
\definecolor{deadblue}{rgb}{0.2, 0.2, 0.7}

%-----------------------------------------------------------------------------------%
% PACOTES DIVERSOS                                                                  %
%-----------------------------------------------------------------------------------%
\usepackage{icomma} % Permite uso de vírgula como separador decimal
\usepackage{url} % Pacote para não ter problemas com URLs. Usar \url{}
%\usepackage{randtext} % Troca a ordem de letras de uma frase (útil com e-mails em
                      % PDFs a serem publicados on-line.
\usepackage[hidelinks]{hyperref}
%\usepackage{showkeys} % Para mostrar o nome dos labels
\usepackage{enumitem} % Facilita o uso de listas, inclusive referências a itens de
                      % listas.
%\usepackage[absolute]{textpos} % Posição absoluta de texto na página
%\usepackage{pdfpages} % Permite incluir documentos em PDF no arquivo
%\usepackage{refcheck} % Verifica as referências procurando por
%                      % labels não usados ou equações numeradas sem labels.
%                      % Verificar o arquivo .log e procurar por RefCheck.
\usepackage[french, onelanguage]{algorithm2e}

%%%%%%%%%%%%%%%%%%%%%%%%%%%%%%%%%%%%%%%%%%%%%%%%%%%%%%%%%%%%%%%%%%%%%%%%%%%%%%%%%%%%%
% CONFIGURAÇÕES                                                                     %
%%%%%%%%%%%%%%%%%%%%%%%%%%%%%%%%%%%%%%%%%%%%%%%%%%%%%%%%%%%%%%%%%%%%%%%%%%%%%%%%%%%%%

%-----------------------------------------------------------------------------------%
% FORMATAÇÃO DO TEXTO                                                               %
%-----------------------------------------------------------------------------------%
%\onehalfspacing % Espaçamento 1 1/2 (definido no pacote setspace)

%-----------------------------------------------------------------------------------%
% DEFINIÇÃO DE AMBIENTES MATEMÁTICOS                                                %
%-----------------------------------------------------------------------------------%
\theoremstyle{plain}
\newtheorem*{prop}{Proposition}

%-----------------------------------------------------------------------------------%
% DEFINIÇÃO DE COMANDOS MATEMÁTICOS                                                 %
%-----------------------------------------------------------------------------------%
%\newcommand*\diff{\mathop{}\!\mathrm{d}}

%\newcommand{\norm}[1]{\left\lVert #1\right\lVert} % Norma
%\newcommand{\abs}[1]{\left\lvert #1\right \rvert} % Valor absoluto
%\newcommand{\floor}[1]{\left\lfloor #1 \right\rfloor} % Arredondar para baixo
%\newcommand{\ceil}[1]{\left\lceil #1 \right\rceil} % Arredondar para cima
\DeclarePairedDelimiter{\ceil}{\lceil}{\rceil}
\DeclarePairedDelimiter{\floor}{\lfloor}{\rfloor}
\DeclarePairedDelimiter{\abs}{\lvert}{\rvert}
\DeclareMathOperator*{\argmax}{argmax}

%-----------------------------------------------------------------------------------%
% NUMERAÇÃO DE ELEMENTOS                                                            %
%-----------------------------------------------------------------------------------%
%\numberwithin{table}{section}
%\numberwithin{table}{subsection}
%\numberwithin{figure}{section}
%\numberwithin{figure}{subsection}
%\numberwithin{equation}{section}
%\numberwithin{equation}{subsection}
%\numberwithin{theo}{chapter}
%\numberwithin{theo}{subsection}

% Maximal percentage of the page occupied by floats
\renewcommand\floatpagefraction{.9}
\renewcommand\topfraction{.9}
\renewcommand\bottomfraction{.9}
\renewcommand\textfraction{.1}
% Maximal number of floats per page
\setcounter{totalnumber}{50}
\setcounter{topnumber}{50}
\setcounter{bottomnumber}{50}

%%%%%%%%%%%%%%%%%%%%%%%%%%%%%%%%%%%%%%%%%%%%%%%%%%%%%%%%%%%%%%%%%%%%%%%%%%%%%%%%%%%%%
% ESTRUTURA DO DOCUMENTO                                                            %
%%%%%%%%%%%%%%%%%%%%%%%%%%%%%%%%%%%%%%%%%%%%%%%%%%%%%%%%%%%%%%%%%%%%%%%%%%%%%%%%%%%%%
\begin{document}

\setlist[itemize]{label=\textbullet, nosep}

\pagestyle{plain}

\title{Projet UE MOGPL --- MU4IN200 \\ Dice Battle}
\author{Ariana Carnielli \\ Ivan Kachaikin}
\date{}

\maketitle


%\tableofcontents

\section{Introduction}

\section{Question 1}

$Q(d,k)$ est la probabilité d'obtenir $k$ points en jetant $d$ dés sachant qu'aucun dé a tombé sur $1$. On numérote les dés de $1$ à $d$ et on note $j$ le résultat du dernier dé, qui est donc un nombre entre $2$ à $6$. Alors on aura $k$ points si et seulement si les $d-1$ premiers dés donnent $k-j$ points. Les valeurs possibles de $j$ étant disjointes, on peut dire que $Q(d, k)$ est la somme pour toutes les valeurs de $j$ de $Q(d-1, k-j)$ multiplié par la probabilité que le dernier dé donne $j$ sachant qu'il n'est pas $1$ (égale à $\frac{1}{5}$), ce qui donne donc~:
\begin{equation}
\label{RecurrenceQ}
Q(d, k) = \sum_{j=2}^6 \frac{Q(d-1, k-j)}{5}.
\end{equation}

Plus précisément, soient $X_1, \dotsc, X_d$ des variables aléatoires indépendantes et distribuées selon une loi uniforme sur $\{1, \dotsc, 6\}$, représentant les résultats de chaque dé. Soient $S_d = \sum_{i=1}^d X_i$ et $S_{d-1} = \sum_{i=1}^{d-1} X_i$. On peut réécrire $Q(d, k)$ comme
\[
Q(d, k) = \mathbb P(S_d = k \mid X_1 \neq 1, \dotsc, X_d \neq 1).
\]

Montrons d'abord que $P(d, k) = \left(\frac{5}{6}\right)^d Q(d, k)$ pour $2d \leq k \leq 6d$. On a
\begin{align*}
P(d, k) & = \mathbb P(S_d = k, X_1 \neq 1, \dotsc, X_d \neq 1) \\
& = \mathbb P(S_d = k \mid X_1 \neq 1, \dotsc, X_d \neq 1) \mathbb P(X_1 \neq 1, \dotsc, X_d \neq 1) \\
& = Q(d, k) \mathbb P(X_1 \neq 1) \dotsm \mathbb P(X_d \neq 1) = Q(d, k) \left(\tfrac{5}{6}\right)^d.
\end{align*}

On montre maintenant la relation \eqref{RecurrenceQ} pour $d \geq 2$ et $2d \leq k \leq 6d$. On a $S_d = S_{d-1} + X_d$. Conditionnellement à $X_1 \neq 1, \dotsc, X_d \neq 1$, $X_d$ ne peut prendre que les valeurs $j$ allant de $2$ à $6$, et, comme les évènements $X_d = 2, \dotsc, X_d = 6$ forment une partition, on a
\begin{align*}
Q(d, k) & = \mathbb P(S_{d-1} + X_d = k \mid X_1 \neq 1, \dotsc, X_d \neq 1) \\
& = \sum_{j=2}^6 \mathbb P(S_{d-1} = k-j, X_d = j \mid X_1 \neq 1, \dotsc, X_d \neq 1) \\
& = \sum_{j=2}^6 \mathbb P(S_{d-1} = k-j \mid X_1 \neq 1, \dotsc, X_d \neq 1)\mathbb P(X_d = j \mid X_1 \neq 1, \dotsc, X_d \neq 1),
\end{align*}
où l'on utilise le fait que $S_{d-1}$ et $X_d$ sont indépendants. Comme $S_{d-1}$ ne dépend pas de $X_d$, on obtient, par la définition de $Q$, que
\[\mathbb P(S_{d-1} = k-j \mid X_1 \neq 1, \dotsc, X_d \neq 1) = \mathbb P(S_{d-1} = k-j \mid X_1 \neq 1, \dotsc, X_{d-1} \neq 1) = Q(d-1, k-j),\]
et en plus $\mathbb P(X_d = j \mid X_1 \neq 1, \dotsc, X_d \neq 1) = \mathbb P(X_d = j \mid X_d \neq 1) = \frac{1}{5}$. Donc
\begin{equation*}
%\label{RecurrenceQ}
Q(d, k) = \sum_{j=2}^6 \frac{Q(d-1, k-j)}{5}.
\end{equation*}

Les cas d'initialisation correspondent à $d = 1$ et $k \in \{2, \dotsc, 6\}$, auquel cas on a $Q(1, k) = \frac{1}{5}$. En plus, $Q(1, k) = 0$ pour $k > 6$.

\section{Question 2}

Remarquons que la formule 
\begin{equation}
\label{EPd}
EP(d) = 4 d \left(\frac{5}{6}\right)^d + 1 - \left(\frac{5}{6}\right)^d
\end{equation}
provient du fait qu'on a une probabilité de $1 - \left(\frac{5}{6}\right)^d$ de ne marquer qu'un seul point et une probabilité de $\left(\frac{5}{6}\right)^d$ de marquer entre $2d$ et $6d$ points, et l'espérance du nombre de points marqué dans ce dernier cas est égale à $4d$.

On veut maximiser $EP(d)$ pour $d \in \{1, \dotsc, D\}$. Pour éviter de faire une recherche exhaustive, on peut faire une étude de la fonction $EP(d)$ en relaxant d'abord $d$ à une variable réelle. Dans ce cas, on calcule
\[
EP^\prime(d) = 4 \left(\frac{5}{6}\right)^d + 4 d \left(\frac{5}{6}\right)^d \ln \frac{5}{6} - \left(\frac{5}{6}\right)^d \ln \frac{5}{6} = \left(\frac{5}{6}\right)^d \left(4 + 4 d \ln \frac{5}{6} - \ln\frac{5}{6}\right).
\]
On cherche les valeurs $d^\ast$ telles que $EP^\prime(d^\ast) = 0$. Cela arrive si et seulement si
\[
4 + 4 d^\ast \ln \frac{5}{6} - \ln\frac{5}{6} = 0
\]
et on calcule alors
\[
d^\ast = \frac{\ln \frac{5}{6} - 4}{4 \ln \frac{5}{6}} = \frac{\ln \frac{6}{5} + 4}{4 \ln \frac{6}{5}} \approx 5,735.
\]
On remarque aussi que $EP^\prime(d) > 0$ pour $d < d^\ast$ et $EP^\prime(d) < 0$ pour $d > d^\ast$, donc $EP$ est strictement croissante sur $\left]-\infty, d^\ast\right[$ et strictement décroissante pour $\left]d^\ast, +\infty\right[$, et atteint ainsi son maximum global à $d = d^\ast$.

On revient maintenant à une variable discrète $d \in \{1, \dotsc, D\}$. Grâce à l'étude précédente, $EP(1) < EP(2) < \dotsb <EP(5)$ et $EP(6) > EP(7) > \dotsb$, ainsi les candidats à maximum global de $EP$ dans les entiers sont $5$ et $6$. On calcule $EP(5) \approx 8,636$ et $EP(6) \approx 8.703$, donc le maximum global est atteint en $d = 6$. Comme $d \in \{1, \dotsc, D\}$, cela n'arrive que lorsque $D \geq 6$~; dans le cas contraire, $EP$ est croissante sur $\{1, \dotsc, D\}$ et le maximum est atteint en $D$. On a donc $d^\ast(D) = \min(D, 6)$. La méthode implémentée retourne donc cette valeur.

 %Si il reste de l'espace, on peut mettre ici le calcul du maximum par dérivée pour montrer qu'il est toujours égal à $\min(D, 6)$.

\section{Question 3}

On considère ici que $D = 3$ et $N = 2$, auquel cas $d^\ast(D) = 3$. On suppose que le joueur 1 choisit de jouer toujours $3$ dés alors que le joueur 2 choisit de jouer toujours un seul dé. Soient $X_i$ et $Y_i$ les gains des joueurs 1 et 2, respectivement, au $i$-ème lancer de dés. Alors $X_i \in \{1, 6, 7, \dotsc, 18\}$ et $Y_i \in \{1, 2, \dotsc, 6\}$. En particulier, on remarque que, si $X_1 > 1$, alors le joueur $1$ est sûr de gagner.

Comme $N = 2$, le jeu se finira au maximum au bout de $2$ tours. Soit $G$ la variable aléatoire donnant le gain final du joueur 1~: $G = 1$ si le joueur 1 gagne et $G = -1$ si le joueur 2 gagne (il est impossible d'avoir un match nul dans la variante séquentielle). Les évènements $G = 1$ et $G = -1$ peuvent être décrits en fonction de $X_i$ et $Y_i$ de la façon suivante~:
\begin{align*}
G = 1 : \quad & X_1 > 1 \cup \bigl(X_1 = 1 \cap Y_1 = 1\bigr) \\
G = -1 : \quad & X_1 = 1 \cap Y_1 > 1
\end{align*}
Cela permet de calculer les probabilités~:
\begin{align*}
\mathbb P(G = 1) & = \left(\frac{5}{6}\right)^3 + \left(1 - \left(\frac{5}{6}\right)^3\right)\frac{1}{6} \approx 0,6489\\
\mathbb P(G = -1) & = \left(1 - \left(\frac{5}{6}\right)^3\right) \frac{5}{6} \approx 0,3511
\end{align*}
On a donc $\mathbb E(G) \approx 0,2978$. Si, à la place de $d^\ast(D) = 3$, le joueur 1 avait choisit de jouer $1$ dé comme le joueur 2, les probabilités de $G = 1$ et $G = -1$ auraient été
\begin{align*}
\mathbb P(G = 1) & = \frac{5}{6} + \left(\frac{1}{6}\right)^2 \approx 0,8611\\
\mathbb P(G = -1) & = \frac{1}{6} \frac{5}{6} \approx 0,1389
\end{align*}
Cela donne $\mathbb E(G) \approx 0,7222$. Ainsi, l'espérance de gain du joueur 1 est plus grande s'il ne choisit de joueur qu'un seul dé dans ce cas.

On remarque que la situation de cet exemple n'est pas exceptionnelle~: le cas $N = 2$ est équivalant, par exemple, au cas $N = 100$ lorsque les deux joueurs sont à égalité avec $98$ points.

\section{Question 4}

Pour faciliter l'analyse, on représente l'état courant du jeu par un triplet $(i, j, n)$ où $i$ et $j$ sont les points cumulés des joueurs 1 et 2, respectivement, et $n \in \{1, 2\}$ indique qui est le prochain joueur à jouer. On note l'espérance de gain du joueur 1 dans l'état $(i, j, n)$ par $EG(i, j, n)$, en supposant que lui-même et son adversaire jouent toujours de façon optimale. On remarque que $EG(i, j, n)$ représente toujours l'espérance de gain du joueur 1, même lorsque $n = 2$.

Dans l'état $(i, j, 1)$, si le joueur 1 décide de jouer $d$ dés et qu'il obtient $k$ points, le prochain état sera $(i+k, j, 2)$. Comme il obtient $k$ points avec probabilité $P(d, k)$, l'espérance de gain du joueur 1 lorsqu'il choisit de jouer $d$ dés (et en supposant que lui-même et le joueur 2 jouent de façon optimale dans la suite) est
\begin{equation}
\label{EsperanceGainDFixe}
\sum_{k=1}^{6d} P(d, k) EG(i+k, j, 2).
\end{equation}
Ainsi, son choix optimal est de choisir le nombre $d$ de dés qui maximise la quantité ci-dessus, ce qui donne
\[
EG(i, j, 1) = \max_{d \in \{1, \dotsc, D\}} \sum_{k=1}^{6d} P(d, k) EG(i+k, j, 2).
\]
Comme le jeu est symétrique par rapport à la permutation des deux joueurs (en changeant le signe du gain du joueur 1), on a $EG(i+k, j, 2) = -EG(j, i+k, 1)$, et ainsi on obtient la formule
\[
EG(i, j, 1) = \max_{d \in \{1, \dotsc, D\}} \left(-\sum_{k=1}^{6d} P(d, k) EG(j, i+k, 1)\right).
\]
Comme cela ne fait intervenir que les espérances de gain lorsque c'est au joueur 1 de jouer (donc $n = 1$), on peut supprimer la troisième composante de l'état de la notation, comme à l'énoncé, pour arriver à
\begin{equation}
\label{EsperanceGainRec}
EG(i, j) = \max_{d \in \{1, \dotsc, D\}} \left(-\sum_{k=1}^{6d} P(d, k) EG(j, i+k)\right),
\end{equation}
où $EG(i, j)$ doit se comprendre comme $EG(i, j, 1)$.

On initialise la récurrence en remarquant que $EG(i, j) = 1$ si $i \geq N$ et $j < N$ et que $EG(i, j) = -1$ si $i < N$ et $j \geq N$. Comme le jeu est séquentiel, il n'est pas nécessaire d'initialiser $EG(i, j)$ pour $i \geq N$ et $j \geq N$~: il est impossible que les deux joueurs aient une quantité supérieure ou égale à $N$ points car le jeu s'arrête dès que le premier joueur atteint $N$ points ou plus.

\section{Question 5}

Notons $OPT(i, j)$ la stratégie optimale du joueur 1 dans l'état $(i, j, 1)$, c'est-à-dire le nombre de dés qu'il doit lancer pour maximiser l'espérance de son gain. Cela revient à maximiser \eqref{EsperanceGainDFixe} par rapport à $d$, ce qui donne
\[
OPT(i, j) = \argmax_{d \in \{1, \dotsc, D\}} \left(-\sum_{k=1}^{6d} P(d, k) EG(j, i+k)\right).
\]
Cela peut être calculé en même temps que le calcul récursif de $EG$ par la formule \eqref{EsperanceGainRec}.

\section{Question 6}

Dans ce cas, la somme dans \eqref{EsperanceGainRec} doit commencer à $k = 0$ à la place de $k = 1$. Le calcul de $EG(i, j)$ par \eqref{EsperanceGainRec} utilise en particulier, dans le second membre, la valeur de $EG(j, i)$. Or, le calcul de $EG(j, i)$ par \eqref{EsperanceGainRec} utilise dans son second membre la valeur de $EG(i, j)$. Ainsi, \eqref{EsperanceGainRec} ne permet pas de calculer $EG(i, j)$ de façon explicite, mais donne uniquement une relation implicite où $EG(i, j)$ dépend de lui-même. Il faudrait alors implémenter une méthode pour être capable de calculer $EG(i, j)$ à partir de cette relation implicite.

\section{Question 10}

Soit $G$ la variable aléatoire donnant le gain du joueur 1 lorsqu'il a jeté $d_1$ dés et le joueur 2 a jeté $d_2$ dés. Alors, par la définition de l'espérance,
\[
EG_1(d_1, d_2) = 1 \cdot \mathbb P(G = 1 \mid d_1, d_2) + 0 \cdot \mathbb P(G = 0 \mid d_1, d_2) + (-1) \cdot \mathbb P(G = -1 \mid d_1, d_2).
\]
Soient $K_1$ et $K_2$ les variables aléatoires représentant le nombre de points obtenus par les joueurs 1 et 2, respectivement. Les évènements $G = 1$, $G = 0$ et $G = -1$ peuvent s'écrire en termes de $K_1$ et $K_2$ comme $K_1 > K_2$, $K_1 = K_2$ et $K_1 < K_2$, respectivement. On a
\[
\{K_1 > K_2 \mid d_1, d_2\} = \bigcup_{j=1}^{6 d_2} \bigcup_{i = j+1}^{6 d_1} \{K_1 = i, K_2 = j \mid d_1, d_2\}.
\]
On remarque que certains des ensembles du membre de droite peuvent être vides, par exemple pour $1 < i < 2 d_1$ ou $1 < j < 2 d_2$. En plus, l'union sur $i$ peut être vide, par exemple dans le cas $j \geq 6 d_1$, auquel cas il n'y a aucun $i$ possible entre $j+1$ et $6 d_1$. On a donc des ensembles vides qui ne changent pas l'union finale. De la même manière, on peut facilement obtenir que
\[
\{K_1 < K_2 \mid d_1, d_2\} = \bigcup_{i=1}^{6 d_1} \bigcup_{j = i+1}^{6 d_2} \{K_1 = i, K_2 = j \mid d_1, d_2\}.
\]
Avant d'obtenir la formule de $EG_1(d_1, d_2)$, on remarque que tous les évènements dans les unions ci-dessus sont deux à deux disjoints, et donc la probabilité de leur union est la somme de leurs probabilités. En plus, comme $K_1$ et $K_2$ sont indépendantes et $K_i$ ne dépend que de $d_i$, on a $\mathbb P(K_1 = i, K_2 = j \mid d_1, d_2) = P(d_1, i) P(d_2, j)$. On peut finalement obtenir une formule pour $EG_1(d_1, d_2)$~:
\begin{align*}
EG_1(d_1, d_2) & = \mathbb P(G = 1 \mid d_1, d_2) - \mathbb P(G = -1 \mid d_1, d_2) \\
& = \sum_{j=1}^{6 d_2} \sum_{i = j+1}^{6 d_1} \mathbb P(K_1 = i, K_2 = j \mid d_1, d_2) - \sum_{i=1}^{6 d_1} \sum_{j = i+1}^{6 d_2} \mathbb P(K_1 = i, K_2 = j \mid d_1, d_2) \\
& = \sum_{j=1}^{6 d_2} \sum_{i = j+1}^{6 d_1} P(d_1, i) P(d_2, j) - \sum_{i=1}^{6 d_1} \sum_{j = i+1}^{6 d_2} P(d_1, i) P(d_2, j) \\
& = \sum_{j=1}^{6 d_2} P(d_2, j) \sum_{i = j+1}^{6 d_1} P(d_1, i) - \sum_{i=1}^{6 d_1} P(d_1, i) \sum_{j = i+1}^{6 d_2} P(d_2, j)
\end{align*}

En utilisant cette formule, on calcule, pour $D = 3$, la matrice de gains
\[
\begin{array}{|c|ccc|}
\hline
d_1 \backslash d_2 & 1 & 2 & 3 \\
\hline
1 & 0     & -0.375 & -0.227 \\
2 & 0.375 &  0     & -0.199 \\
3 & 0.227 &  0.199 &  0 \\
\hline
\end{array}
\]
Comme le jeu est à somme nulle, la matrice est anti-symétrique.

\section{Question 11}

Si on suppose que le joueur 2 connait la probabilité $p_1(i)$, $i = 1, 2, \dotsc, D$, alors il peut calculer le gain espéré du joueur 1. En fait, quand le joueur 2 joue $j$ dés ($j = 1, 2, \dotsc, D$), le joueur 1 gagne en espérance $EG_1(1, j)$ avec probabilité $p_1(1)$, $EG_1(2, j)$ avec probabilité $p_1(2)$, $\dotsc$, $EG_1(D, j)$ avec probabilité $p_1(D)$. L'espérance du gain du joueur 1 est donc donnée par la formule suivante~:
\[\sum_{i=1}^D p_1(i) EG_1(i, j),\]
quand le joueur 2 joue $j$ dés. Par conséquent, il ne reste au joueur 2 qu'à minimiser cette espérance de gain. On suppose qu'il choisit une stratégie mixte $p_2(1), p_2(2), \dotsc, p_2(D)$ (on remarque que les stratégies mixtes contiennent les pures). Le joueur 2 perdra ainsi $\sum_{i=1}^D p_1(i) EG_1(i, 1)$ avec probabilité $p_2(1)$, $\sum_{i=1}^D p_1(i) EG_1(i, 2)$ avec probabilité $p_2(2)$, $\dotsc$, $\sum_{i=1}^D p_1(i) EG_1(i, D)$ avec probabilité $p_2(D)$. Ainsi, l'espérance de sa perte est
\[
\sum_{j=1}^D p_2(j) \sum_{i=1}^D p_1(i) EG_1(i, j) = \sum_{i=1}^D \sum_{j=1}^D p_1(i) EG_1(i, j) p_2(j).
\]
Notons $p_1 = \bigl(p_1(1), p_1(2), \dotsc, p_1(D)\bigr)$, $p_2 = \bigl(p_2(1), p_2(2), \dotsc, p_2(D)\bigr)$, et $EG_1$ la matrice dont l'élément à la ligne $i$ et colonne $j$ est $EG_1(i, j)$, $i, j = 1, 2, \dotsc, D$. La perte moyenne du joueur 2 s'écrit donc~:
\[
p_1^\top EG_1 p_2,
\]
où on a supposé que $p_1$ et $p_2$ sont des vecteurs colonne.

Pour le joueur 2, il faut donc minimiser son espérance de perte. C'est pourquoi il prendra une stratégie mixte $p_2^\ast = \bigl(p_2^\ast(1), p_2^\ast(2), \dotsc, p_2^\ast(D)\bigr)$ telle que
\begin{equation}
\label{OptimalJ2}
p_1^\top EG_1 p_2^\ast = \min_{p_2} p_1^\top EG_1 p_2.
\end{equation}

\section{Question 12}

Le joueur 1 veut maximiser de son côté son espérance de gain. Supposant que le joueur 2 joue optimalement, le joueur 1 peut déduire que, pour toute stratégie $p_1$ qu'il choisit, son adversaire jouera selon \eqref{OptimalJ2}. Sachant ce fait, il doit donc maximiser \eqref{OptimalJ2} de son côté, c'est-à-dire, il doit choisir une stratégie mixte $p_1^\ast = \bigl(p_1^\ast(1), p_1^\ast(2), \dotsc, p_1^\ast(D)\bigr)$ telle que
\begin{equation}
\label{OptimalJ1}
\min_{p_2} {p_1^\ast}^\top EG_1 p_2 = \max_{p_1} \min_{p_2} p_1^\top EG_1 p_2.
\end{equation}
Le joueur 1 doit donc résoudre le problème de maximisation à droite dans \eqref{OptimalJ1} pour déterminer une stratégie $p_1^\ast$ qui maximisera son espérance de gain sachant que son adversaire joue optimalement.

Pour trouver une solution du problème \eqref{OptimalJ1}, on remarque que, comme vu en cours (Lemme~1 du Cours~4),
\[
\max_{p_1} \min_{p_2} p_1^\top EG_1 p_2 = \max_{p_1} \min_{j} p_1^\top EG_1^j,
\]
où $EG_1^j$ est la colonne d'indice $j$ de la matrice $EG_1$ ($j = 1, 2, \dotsc, D$). En plus, d'après les propriétés vues en cours de ce type de problème linéaire, le problème $\max_{p_1} \min_{j} p_1^\top EG_1^j$ peut être résolu en résolvant
\begin{equation}
\label{ProblemeLineaire}
\begin{aligned}
& \max\; z^\prime - z^{\prime\prime} \\
& \begin{aligned}
- p_1^\top EG_1^j + z^\prime - z^{\prime\prime} & \leq 0 & \quad & \forall j \in \{1, \dotsc, D\}, \\
\textstyle\sum_{i=1}^D p_1(i) & \leq 1 \\
\textstyle-\sum_{i=1}^D p_1(i) & \leq -1 \\
z^\prime \geq 0,\; z^{\prime\prime}\geq 0,\; p_1(i) & \geq 0 & & \forall i \in \{1, \dotsc, D\}
\end{aligned}
\end{aligned}
\end{equation}
Donc, en résolvant le problème \eqref{ProblemeLineaire} on peut obtenir un vecteur $p_1^\ast$ qui sera aussi une solution du problème \eqref{OptimalJ1}. Finalement, le joueur 1 prend donc $p_1^\ast$ comme sa stratégie optimale.

\end{document}
