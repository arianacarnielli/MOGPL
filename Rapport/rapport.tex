%%%%%%%%%%%%%%%%%%%%%%%%%%%%%%%%%%%%%%%%%%%%%%%%%%%%%%%%%%%%%%%%%%%%%%%%%%%%%%%%%%%%%
% PACOTES                                                                           %
%%%%%%%%%%%%%%%%%%%%%%%%%%%%%%%%%%%%%%%%%%%%%%%%%%%%%%%%%%%%%%%%%%%%%%%%%%%%%%%%%%%%%
\documentclass[a4paper,11pt]{amsart}

%-----------------------------------------------------------------------------------%
% LAYOUT DA PÁGINA                                                                  %
%-----------------------------------------------------------------------------------%
\usepackage[top=2.25cm, bottom=2.25cm, left=2.25cm, right=2.25cm]{geometry}
%\usepackage{fancyhdr} % Permite controlar como são exibidos os cabeçalhos

%-----------------------------------------------------------------------------------%
% FORMATAÇÃO DO TEXTO                                                               %
%-----------------------------------------------------------------------------------%
%\usepackage{setspace} % Permite definir o espaçamento entre linhas

%-----------------------------------------------------------------------------------%
% PACOTES DE IMAGENS                                                                %
%-----------------------------------------------------------------------------------%
\usepackage[pdftex]{graphicx}
\pdfsuppresswarningpagegroup=1 % A warning issued when several PDF images are
% imported in the same page. Mostly harmless, can be almost always supressed.
%\usepackage[pstarrows]{pict2e} % Amplia as funcionalidades do ambiente picture
\usepackage{tikz}
\usetikzlibrary{shapes, arrows, arrows.meta}

%-----------------------------------------------------------------------------------%
% PACOTES DE TABELAS                                                                %
%-----------------------------------------------------------------------------------%
\usepackage{array} % Facilita a formatação de tabelas
%\usepackage{multirow} % Permite criar células que ocupam várias linhas em uma tabela
\usepackage{longtable} % Permite criar tabelas que quebram de página

%-----------------------------------------------------------------------------------%
% PACOTES MATEMÁTICOS DE BASE                                                       %
%-----------------------------------------------------------------------------------%
\usepackage{amsfonts,amstext,amscd,bezier,amsthm,amssymb}
\usepackage[centertags]{amsmath}

%-----------------------------------------------------------------------------------%
% PACOTES DE SÍMBOLOS MATEMÁTICOS                                                   %
%-----------------------------------------------------------------------------------%
\usepackage{mathtools} % Símbolos matemáticos extras. (ex.: \xrightharpoon)
%\usepackage[integrals]{wasysym} % Muda o estilo das integrais, além de outros
%                                 símbolos extras
%\usepackage[nice]{nicefrac} % Permite o uso de frações "melhores". Usar \nicefrac{}{}

%-----------------------------------------------------------------------------------%
% PACOTES DE FONTES MATEMÁTICAS                                                     %
%-----------------------------------------------------------------------------------%
%\usepackage{mathbbol} % Quase todos os símbolos com \mathbb
%\usepackage{bbm} % Extensão dos símbolos de \mathbb. Usar comando \mathbbm
%\usepackage{calrsfs} % Muda o estilo de \mathcal
%\usepackage[mathcal]{euscript} % Muda o estilo de \mathcal

%-----------------------------------------------------------------------------------%
% PACOTES DE CODIFICAÇÃO DE FONTES                                                  %
%-----------------------------------------------------------------------------------%
\usepackage[utf8]{inputenc} % Permite o uso de caracteres ISO 8859-1, incluindo os
%                               caracteres acentuados diretamente.
\usepackage[T1]{fontenc} % Uso de fontes T1, necessário para tratar caracteres
%                          acentuados como um único bloco.

%-----------------------------------------------------------------------------------%
% PACOTES DE LÍNGUAS                                                                %
%-----------------------------------------------------------------------------------%
\usepackage[french]{babel} % Seleciona a língua do documento, definindo nomes de
%                              seções, nome do índice, da bibliografia, etc. Em caso
%                              de documento com mais de uma língua, a padrão é a
%                              última.
\NoAutoSpaceBeforeFDP % Utilizar em francês se quiser evitar espaços antes de :

%-----------------------------------------------------------------------------------%
% PACOTES DE BIBLIOGRAFIA                                                           %
%-----------------------------------------------------------------------------------%
%\usepackage{babelbib} % Permite definir a língua das entradas da bibliografia. Usar
%                       [fixlanguage] para uma mesma língua para todas as entradas e
%                       \selectbiblanguage{} para definir a língua. Um estilo compa-
%                       tível com babelbib deve ser usado (ex: babplain)
\usepackage{cite} % Organiza os elementos citados dentro de um mesmo \cite.

%-----------------------------------------------------------------------------------%
% PACOTES DE FONTES                                                                 %
%-----------------------------------------------------------------------------------%
% Computer Modern (fonte padrão)                                                    %
% - - - - - - - - - - - - - - - - - - - - - - - - - - - - - - - - - - - - - - - - - %
%\usepackage{ae} % A usar com a fonte padrão do LaTeX quando forem gerados PDFs, para
%                 corrigir erros de visualização

% Computer Modern Bright (sans serif)                                               %
% - - - - - - - - - - - - - - - - - - - - - - - - - - - - - - - - - - - - - - - - - %
%\usepackage{cmbright}

% Times New Roman                                                                   %
% - - - - - - - - - - - - - - - - - - - - - - - - - - - - - - - - - - - - - - - - - %
%\usepackage{mathptmx} % Muda texto e modo matemático
%\usepackage{times} % Apenas texto, não muda modo matemático

% Arial                                                                             %
% - - - - - - - - - - - - - - - - - - - - - - - - - - - - - - - - - - - - - - - - - %
%\usepackage[scaled]{uarial} % Arial como fonte sans serif padrão

% Palatino                                                                          %
% - - - - - - - - - - - - - - - - - - - - - - - - - - - - - - - - - - - - - - - - - %
%\usepackage{mathpazo} % Muda texto e modo matemático
%\usepackage{palatino} % Apenas texto, não muda modo matemático

% Concrete                                                                          %
% - - - - - - - - - - - - - - - - - - - - - - - - - - - - - - - - - - - - - - - - - %
%\usepackage{ccfonts} % Texto: Concrete; Matemático: Concrete Math
%\usepackage{ccfonts, eulervm} % Texto: Concrete; Matemático: Euler

% Iwona                                                                             %
% - - - - - - - - - - - - - - - - - - - - - - - - - - - - - - - - - - - - - - - - - %
%\usepackage[math]{iwona} % Texto e modo matemático: Iwona

% Kurier                                                                            %
% - - - - - - - - - - - - - - - - - - - - - - - - - - - - - - - - - - - - - - - - - %
%\usepackage[math]{kurier} % Texto e modo matemático: Kurier

% Antykwa Póltawskiego                                                              %
% - - - - - - - - - - - - - - - - - - - - - - - - - - - - - - - - - - - - - - - - - %
%\usepackage{antpolt} % Texto: Antykwa Póltawskiego; Matemático: nenhum
                     % Usar fontenc = QX ou OT4

% Utopia                                                                            %
% - - - - - - - - - - - - - - - - - - - - - - - - - - - - - - - - - - - - - - - - - %                     
%\usepackage{fourier} % Texto: Utopia; Matemático: Fourier

% KP Serif                                                                          %
% - - - - - - - - - - - - - - - - - - - - - - - - - - - - - - - - - - - - - - - - - %
\usepackage{kpfonts}

%-----------------------------------------------------------------------------------%
% CORES                                                                             %
%-----------------------------------------------------------------------------------%
\usepackage{color}
\definecolor{darkgreen}{rgb}{0,0.5,0}
\definecolor{darkmagenta}{rgb}{0.5,0,0.5}
\definecolor{darkgray}{rgb}{0.5,0.5,0.5}
\definecolor{darkblue}{rgb}{0.2,0.2,0.4}
\definecolor{darkred}{rgb}{0.6,0.15,0.15}
\definecolor{gray}{rgb}{0.65,0.65,0.65}
\definecolor{lightgray}{rgb}{0.8,0.8,0.8}
\definecolor{lightblue}{rgb}{0.5,0.5,1}
\definecolor{lightgreen}{rgb}{0.5,1,0.5}
\definecolor{deadred}{rgb}{0.7, 0.2, 0.2}
\definecolor{deadblue}{rgb}{0.2, 0.2, 0.7}

%-----------------------------------------------------------------------------------%
% PACOTES DIVERSOS                                                                  %
%-----------------------------------------------------------------------------------%
\usepackage{icomma} % Permite uso de vírgula como separador decimal
\usepackage{url} % Pacote para não ter problemas com URLs. Usar \url{}
%\usepackage{randtext} % Troca a ordem de letras de uma frase (útil com e-mails em
                      % PDFs a serem publicados on-line.
\usepackage[hidelinks]{hyperref}
%\usepackage{showkeys} % Para mostrar o nome dos labels
\usepackage{enumitem} % Facilita o uso de listas, inclusive referências a itens de
                      % listas.
%\usepackage[absolute]{textpos} % Posição absoluta de texto na página
%\usepackage{pdfpages} % Permite incluir documentos em PDF no arquivo
%\usepackage{refcheck} % Verifica as referências procurando por
%                      % labels não usados ou equações numeradas sem labels.
%                      % Verificar o arquivo .log e procurar por RefCheck.
\usepackage[french, onelanguage]{algorithm2e}

%%%%%%%%%%%%%%%%%%%%%%%%%%%%%%%%%%%%%%%%%%%%%%%%%%%%%%%%%%%%%%%%%%%%%%%%%%%%%%%%%%%%%
% CONFIGURAÇÕES                                                                     %
%%%%%%%%%%%%%%%%%%%%%%%%%%%%%%%%%%%%%%%%%%%%%%%%%%%%%%%%%%%%%%%%%%%%%%%%%%%%%%%%%%%%%

%-----------------------------------------------------------------------------------%
% FORMATAÇÃO DO TEXTO                                                               %
%-----------------------------------------------------------------------------------%
%\onehalfspacing % Espaçamento 1 1/2 (definido no pacote setspace)

%-----------------------------------------------------------------------------------%
% DEFINIÇÃO DE AMBIENTES MATEMÁTICOS                                                %
%-----------------------------------------------------------------------------------%
\theoremstyle{plain}
\newtheorem*{prop}{Proposition}

%-----------------------------------------------------------------------------------%
% DEFINIÇÃO DE COMANDOS MATEMÁTICOS                                                 %
%-----------------------------------------------------------------------------------%
%\newcommand*\diff{\mathop{}\!\mathrm{d}}

%\newcommand{\norm}[1]{\left\lVert #1\right\lVert} % Norma
%\newcommand{\abs}[1]{\left\lvert #1\right \rvert} % Valor absoluto
%\newcommand{\floor}[1]{\left\lfloor #1 \right\rfloor} % Arredondar para baixo
%\newcommand{\ceil}[1]{\left\lceil #1 \right\rceil} % Arredondar para cima
\DeclarePairedDelimiter{\ceil}{\lceil}{\rceil}
\DeclarePairedDelimiter{\floor}{\lfloor}{\rfloor}
\DeclarePairedDelimiter{\abs}{\lvert}{\rvert}
\DeclareMathOperator*{\argmax}{argmax}

%-----------------------------------------------------------------------------------%
% NUMERAÇÃO DE ELEMENTOS                                                            %
%-----------------------------------------------------------------------------------%
%\numberwithin{table}{section}
%\numberwithin{table}{subsection}
%\numberwithin{figure}{section}
%\numberwithin{figure}{subsection}
%\numberwithin{equation}{section}
%\numberwithin{equation}{subsection}
%\numberwithin{theo}{chapter}
%\numberwithin{theo}{subsection}

% Maximal percentage of the page occupied by floats
\renewcommand\floatpagefraction{.9}
\renewcommand\topfraction{.9}
\renewcommand\bottomfraction{.9}
\renewcommand\textfraction{.1}
% Maximal number of floats per page
\setcounter{totalnumber}{50}
\setcounter{topnumber}{50}
\setcounter{bottomnumber}{50}

%%%%%%%%%%%%%%%%%%%%%%%%%%%%%%%%%%%%%%%%%%%%%%%%%%%%%%%%%%%%%%%%%%%%%%%%%%%%%%%%%%%%%
% ESTRUTURA DO DOCUMENTO                                                            %
%%%%%%%%%%%%%%%%%%%%%%%%%%%%%%%%%%%%%%%%%%%%%%%%%%%%%%%%%%%%%%%%%%%%%%%%%%%%%%%%%%%%%
\begin{document}

\setlist[itemize]{label=\textbullet, nosep}

\pagestyle{plain}

\title{Projet UE MOGPL --- MU4IN200 \\ Dice Battle}
\author{Ariana Carnielli}
\date{}

\maketitle


%\tableofcontents

\section{Introduction}

\section{Question 1}

$Q(d,k)$ est la probabilité d'obtenir $k$ points en jetant $d$ dés sachant qu'aucun dé a tombé sur $1$. On numérote les dés de $1$ à $d$ et on note $j$ le résultat du dernier dé, qui est donc un nombre entre $2$ à $6$. Alors on aura $k$ points si et seulement si les $d-1$ premiers dés donnent $k-j$ points. Les valeurs possibles de $j$ étant disjointes, on peut dire que $Q(d, k)$ est la somme pour toutes les valeurs de $j$ de $Q(d-1, k-j)$ multiplié par la probabilité que le dernier dé donne $j$ sachant qu'il n'est pas $1$ (égale à $\frac{1}{5}$), ce qui donne donc~:
\[
Q(d, k) = \sum_{j=2}^6 \frac{Q(d-1, k-j)}{5}.
\]

Plus précisément, soient $X_1, \dotsc, X_d$ des variables aléatoires indépendantes et distribuées selon une loi uniforme sur $\{1, \dotsc, 6\}$, représentant les résultats de chaque dé. Soient $S_d = \sum_{i=1}^d X_i$ et $S_{d-1} = \sum_{i=1}^{d-1} X_i$. On peut réécrire $Q(d, k)$ comme
\[
Q(d, k) = \mathbb P(S_d = k \mid X_1 \neq 1, \dotsc, X_d \neq 1).
\]
Or $S_d = S_{d-1} + X_d$. Conditionnellement à $X_1 \neq 1, \dotsc, X_d \neq 1$, $X_d$ ne peut prendre que les valeurs $j$ allant de $2$ à $6$, et, comme les évènements $X_d = 2, \dotsc, X_d = 6$ forment une partition, on a
\begin{align*}
Q(d, k) & = \mathbb P(S_{d-1} + X_d = k \mid X_1 \neq 1, \dotsc, X_d \neq 1) \\
& = \sum_{j=2}^6 \mathbb P(S_{d-1} = k-j, X_d = j \mid X_1 \neq 1, \dotsc, X_d \neq 1) \\
& = \sum_{j=2}^6 \mathbb P(S_{d-1} = k-j \mid X_1 \neq 1, \dotsc, X_d \neq 1)\mathbb P(X_d = j \mid X_1 \neq 1, \dotsc, X_d \neq 1),
\end{align*}
où l'on utilise le fait que $S_{d-1}$ et $X_d$ sont indépendants. Comme $S_{d-1}$ ne dépend pas de $X_d$, on obtient, par la définition de $Q$, que
\[\mathbb P(S_{d-1} = k-j \mid X_1 \neq 1, \dotsc, X_d \neq 1) = \mathbb P(S_{d-1} = k-j \mid X_1 \neq 1, \dotsc, X_{d-1} \neq 1) = Q(d-1, k-j),\]
et en plus $\mathbb P(X_d = j \mid X_1 \neq 1, \dotsc, X_d \neq 1) = \mathbb P(X_d = j \mid X_d \neq 1) = \frac{1}{5}$. Donc
\begin{equation}
\label{RecurrenceQ}
Q(d, k) = \sum_{j=2}^6 \frac{Q(d-1, k-j)}{5}.
\end{equation}

Les cas d'initialisation correspondent à $d = 1$ et $k \in \{2, \dotsc, 6\}$, auquel cas on a $Q(1, k) = \frac{1}{5}$. En plus, $Q(1, k) = 0$ pour $k > 6$.

%\section{Question 2}

% Si il reste de l'espace, on peut mettre ici le calcul du maximum par dérivée pour montrer qu'il est toujours égal à $\min(D, 6)$.

\section{Question 3}

On considère ici que $D = 10$ et $N = 2$, auquel cas $d^\ast(D) = 6$. On suppose que le joueur 1 choisit de jouer toujours $6$ dés alors que le joueur 2 choisit de jouer toujours un seul dé. Soient $X_i$ et $Y_i$ les gains des joueurs 1 et 2, respectivement, au $i$-ème lancer de dés. Alors $X_i \in \{1, 12, 13, \dotsc, 36\}$ et $Y_i \in \{1, 2, \dotsc, 6\}$. En particulier, on remarque que, si $X_1 > 1$, alors le joueur $1$ est sûr de gagner.

Comme $N = 2$, le jeu se finira au maximum au bout de $2$ tours. Soit $G$ la variable aléatoire donnant le gain final du joueur 1~: $G = 1$ si le joueur 1 gagne, $G = 0$ si on a un match nul et $G = -1$ si le joueur 2 gagne. Les évènements $G = 1$, $G = 0$ et $G = -1$ peuvent être décrits en fonction de $X_i$ et $Y_i$ de la façon suivante~:
\begin{align*}
G = 1 : \quad & X_1 > 1 \cup \bigl(X_1 = 1 \cap Y_1 = 1 \cap X_2 > 1\bigr) \\
G = 0 : \quad & X_1 = 1 \cap Y_1 = 1 \cap X_2 = 1 \cap Y_2 = 1 \\
G = -1 : \quad & \bigl(X_1 = 1 \cap Y_1 > 1\bigr) \cup \bigl(X_1 = 1 \cap Y_1 = 1 \cap X_2 = 1 \cap Y_2 > 1\bigr)
\end{align*}
Cela permet de calculer les probabilités~:
\begin{align*}
\mathbb P(G = 1) & = \left(\frac{5}{6}\right)^6 + \left(1 - \left(\frac{5}{6}\right)^6\right)\frac{1}{6} \left(\frac{5}{6}\right)^6 \approx 0.3720\\
\mathbb P(G = 0) & = \left(1 - \left(\frac{5}{6}\right)^6\right)^2 \left(\frac{1}{6}\right)^2 \approx 0.0123\\
\mathbb P(G = -1) & = \left(1 - \left(\frac{5}{6}\right)^6\right) \frac{5}{6} + \left(1 - \left(\frac{5}{6}\right)^6\right)^2 \frac{1}{6} \frac{5}{6} \approx 0.6157
\end{align*}
Comme $\mathbb P(G = 1) < \mathbb P(G = -1)$, on aura $\mathbb E(G) < 0$, le joueur 1 a une espérance de gain négative. Néanmoins, si le joueur 1 choisit de jouer comme le joueur 2 et ne lancer qu'un dé, alors, par la symétrie entre les joueurs dans ce cas, on aura $\mathbb P(G = 1) = \mathbb P(G = -1)$ et donc $\mathbb E(G) = 0$. Il est donc préférable dans ce cas pour le joueur 1 de jouer 1 dé que de jouer $d^\ast(10) = 6$ dés.

On remarque que la situation de cet exemple n'est pas exceptionnelle~: le cas $N = 2$ est équivalant, par exemple, au cas $N = 100$ lorsque les deux joueurs sont à égalité avec $98$ points.

\section{Question 4}

Pour faciliter l'analyse, on représente l'état courant du jeu par un triplet $(i, j, n)$ où $i$ et $j$ sont les points cumulés des joueurs 1 et 2, respectivement, et $n \in \{1, 2\}$ indique qui est le prochain joueur à jouer. On note l'espérance de gain du joueur 1 dans l'état $(i, j, n)$ par $EG(i, j, n)$, en supposant que lui-même et son adversaire jouent toujours de façon optimale. On remarque que $EG(i, j, n)$ représente toujours l'espérance de gain du joueur 1, même lorsque $n = 2$.

Dans l'état $(i, j, 1)$, si le joueur 1 décide de jouer $d$ dés et qu'il obtient $k$ points, le prochain état sera $(i+k, j, 2)$. Comme il obtient $k$ points avec probabilité $P(d, k)$, l'espérance de gain du joueur 1 lorsqu'il choisit de jouer $d$ dés (et en supposant que lui-même et le joueur 2 jouent de façon optimale dans la suite) est
\begin{equation}
\label{EsperanceGainDFixe}
\sum_{k=1}^{6d} P(d, k) EG(i+k, j, 2).
\end{equation}
Ainsi, son choix optimal est de choisir le nombre $d$ de dés qui maximise la quantité ci-dessus, ce qui donne
\[
EG(i, j, 1) = \max_{d \in \{1, \dotsc, D\}} \sum_{k=1}^{6d} P(d, k) EG(i+k, j, 2).
\]
Comme le jeu est symétrique par rapport à la permutation des deux joueurs (en changeant le signe du gain du joueur 1), on a $EG(i+k, j, 2) = -EG(j, i+k, 1)$, et ainsi on obtient la formule
\[
EG(i, j, 1) = \max_{d \in \{1, \dotsc, D\}} \left(-\sum_{k=1}^{6d} P(d, k) EG(j, i+k, 1)\right).
\]
Comme cela ne fait intervenir que les espérances de gain lorsque c'est au joueur 1 de jouer (donc $n = 1$), on peut supprimer la troisième composante de l'état de la notation, comme à l'énoncé, pour arriver à
\begin{equation}
\label{EsperanceGainRec}
EG(i, j) = \max_{d \in \{1, \dotsc, D\}} \left(-\sum_{k=1}^{6d} P(d, k) EG(j, i+k)\right),
\end{equation}
où $EG(i, j)$ doit se comprendre comme $EG(i, j, 1)$.

On initialise la récurrence en remarquant que $EG(i, j) = 1$ si $i \geq N$ et $j < N$ et que $EG(i, j) = -1$ si $i < N$ et $j \geq N$. Comme le jeu est séquentiel, il n'est pas nécessaire d'initialiser $EG(i, j)$ pour $i \geq N$ et $j \geq N$~: il est impossible que les deux joueurs aient une quantité supérieure ou égale à $N$ points car le jeu s'arrête dès que le premier joueur atteint $N$ points ou plus.

\section{Question 5}

Notons $OPT(i, j)$ la stratégie optimale du joueur 1 dans l'état $(i, j, 1)$, c'est-à-dire le nombre de dés qu'il doit lancer pour maximiser l'espérance de son gain. Cela revient à maximiser \eqref{EsperanceGainDFixe} par rapport à $d$, ce qui donne
\[
OPT(i, j) = \argmax_{d \in \{1, \dotsc, D\}} \left(-\sum_{k=1}^{6d} P(d, k) EG(j, i+k)\right).
\]
Cela peut être calculé en même temps que le calcul récursif de $EG$ par la formule \eqref{EsperanceGainRec}.

\section{Question 6}

Dans ce cas, la somme dans \eqref{EsperanceGainRec} doit commencer à $k = 0$ à la place de $k = 1$. Le calcul de $EG(i, j)$ par \eqref{EsperanceGainRec} utilise en particulier, dans le second membre, la valeur de $EG(j, i)$. Or, le calcul de $EG(j, i)$ par \eqref{EsperanceGainRec} utilise dans son second membre la valeur de $EG(i, j)$. Ainsi, \eqref{EsperanceGainRec} ne permet pas de calculer $EG(i, j)$ de façon explicite, mais donne uniquement une relation implicite où $EG(i, j)$ dépend de lui-même. Il faudrait alors implémenter une méthode pour être capable de calculer $EG(i, j)$ à partir de cette relation implicite.

\end{document}
